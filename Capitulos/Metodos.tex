\chapter{Métodos}

% O que será feito, como será feito, … para abordar o problema

Este trabalho prevê o desenvolvimento da ferramenta de documentação de implementação funcionalidades. Nela, será possível vincular recursos do repositório do projeto no GitHub (issues, commits, etc) à requisitos e suas implementações, bem como outros referenciais (links para perguntas no Stack Overflow, desenhos, por exemplo) de forma a gerar um guia ou tutorial de como realizar tal implementação novamente no futuro.

A empresa 4Soft terá participação significativa em todo o estudo, desde a concepção até no uso da ferramenta.

% metodo: entrevista, questionário de satisfacao. Focar em pensar em como responder as perguntas

Os procedimentos para o estudo são os seguintes:

\section {Estudo de aplicações existentes}

Será feita uma busca por aplicações semelhantes. Suas limitações serão analisadas de modo que a ferramenta proposta as supra.

\section {Inquérito contextual}

Entrevistas e seções de brainstorm serão feitas com os participantes da empresa júnior mencionada. Será analisado como se dá seu processo de trabalho e como pode se dar o fluxo de atividades na ferramenta através de entrevistas e aplicação de questionários de satisfação.

\section {Sessões de interpretação da equipe}

Reuniões com a equipe de pesquisadores da UFRN que trabalharão no projeto serão feitas para definir o escopo da ferramenta, bem como seu se dará seu design e implementação

\section {Prototipação e implementação da ferramenta}

Nesta etapa, inicialmente, protótipos de baixa fidelidade serão elaborados. Posteriormente, serão expostos a todos os participantes e seu feedback será colhido e analisado. A partir daí, protótipos de maior fidelidade serão elaborados de maneira iterativa e incremental.

\section {Implantação e observação do uso da ferramenta}

A ferramenta então estará disponível para uso nos contextos descritos na seção anterior deste trabalho. A adesão dos desenvolvedores a ferramenta terá analisada nesta etapa, bem como seu uso (quantidade de artefatos de documentação criados, por exemplo) monitorado.

%resposta das peguntas 1 e 2


\section {Avaliação da ferramenta}

Ao fim do período anterior, uma nova bateria de entrevistas e aplicação de questionários de satisfação serão realizadas para avaliar qualitativamente como se deu a utilização da ferramenta, bem como se deram os efeitos de seu uso. 

%resposta das peguntas 3 e 4

A versão final do trabalho será então redigida.
