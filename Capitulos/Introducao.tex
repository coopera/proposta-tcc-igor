\chapter{Introdução}

% A introdução é a parte inicial do texto e que possibilita uma visão geral de todo o trabalho, devendo constar a delimitação do assunto tratado, objetivos da pesquisa, motivação para o desenvolvimento da mesma e outros elementos necessários para situar o tema do trabalho.

% \section{Organização do trabalho}

% Nesta seção deve ser apresentado como está organizado o trabalho, sendo descrito, portanto, do que trata cada capítulo.

% Introdução

%caracterização do problema

% ORIGINAL MEU
% No contexto de equipes de desenvolvimento de software, é constante a troca de conhecimento entre seus participantes.\ignore{FONTE} Ainda mais comum que isso é a implementação de funcionalidades semelhantes em diferentes projetos.\ignore{FONTE}

No contexto de trabalhadores do conhecimento, como na área de desenvolvimento de software, é diária a troca de conhecimento com o objetivo de se obter um melhor desempenho para a organização inteira~\cite{Druker1993}~\cite{Wiig2003}.\ignore{FONTE 1} A gerência de conhecimento lida com o reuso de conhecimento em  suas diferentes formas, como: design de código, requisitos, modelos, dados, padrões e lições aprendidas~\cite{Levy2009}\ignore{FONTE 2}. Dentre esse reuso, se encontra, por exemplo, como se deu a implementação de uma determinada funcionalidade em um determinado projeto de software.

% FONTE 1 - Knowledge workers are required to improve their work on a daily basis in a process that cumulates into a significant improvement in performance for the entire enterprise [6][21].

% FONTE 2 - KM is comprised of the elicitation, packaging and management, and reuse of knowledge in all of its different forms, and in particular, software engineering artifacts as code, design, requirements, models, data, standards, and lessons learned.

%Software development can be improved by recognizing the related knowledge content and structure as well as the required knowledge, and performing planning activities. - 05071412.pdf


Atualmente, as maneiras mais comuns de troca de informações entre desenvolvedores são via oral, escrita ou repasse de referências (documentação, links externos, etc)~\cite{Storey2014}~\cite{Olson2000}~\cite{CubraniC2004}\ignore{FONTE 3} e, em alguns casos, pela natureza do meio, não se mantém registrada para consulta~\cite{Olson2000}.

% FONTE 3 - p100-storey.pdf PROFESSOR: Isso tá ao longo do artigo, como citar?

É comum também desenvolvedores implementarem funcionalidades semelhantes em diferentes contextos usando abordagens ad-hoc~\cite{SangMok2011}. Nesse caso, desenvolvedores mais experientes em um determinado projeto tendem a atuar como mentores~\cite{CubraniC2004} e tal ato, como atividade de gerência de conhecimento, tende a consumir recursos~\cite{Wiig2003}.\ignore{FONTE 4}

% FONTE 4 - One of the cornerstones of KM is improving productivity by effective sharing and transfer of knowledge, which tends to be time-consuming and often impossible [21] - 05071412.pdf.

A elaboração de uma ferramenta capaz de agregar referências de código e tarefas a soluções pode trazer enormes benefícios a equipes de desenvolvimento~\cite{CubraniC2004}, visto que a rotatividade de membros nesse tipo de equipe geralmente está associada ao sucesso ou fracasso de projetos~\cite{Hall2008}. 

Este estudo propõe a elaboração de tal ferramenta com a participação de uma equipe real de desenvolvimento de software. Esta equipe, posteriormente, fará o uso do produto elaborado em seu contexto de trabalho.

A equipe em questão é a 4Soft, empresa júnior dos cursos de Engenharia de Software e Tecnologia da Informação da Universidade Federal do Rio Grande do Norte (UFRN). A empresa atua na área de desenvolvimento de software web para clientes de diversos ramos e é formada exclusivamente por alunos dos cursos de Bacharelado em Engenharia de Software e Bacharelado em Tecnologia da Informação da UFRN~\cite{4Soft}.

O estudo também prevê uma análise crítica dos impactos do uso na mesma no contexto de empresa.




% In software development, it is common for a member of the team to implement a feature very similar to one that was already done by a teammate. In some cases, that is a cumbersome task. Searching through documentation and code, as well as asking other teammates can be time and resource demanding.

% We propose a tool that will permit a team to store documentation for features via GitHub issues and comm its merged with other information like Stack Overflow questions, framework or language documentation or even notes written by teammates. This tool will be designed, tested and evaluated by the developers of 4Soft, the Junior Enterprise of Software Engineer of UFRN.
