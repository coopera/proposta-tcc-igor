\chapter{Objetivos}

% gerais e específicos
Este trabalho tem por finalidade analisar que influência uma ferramenta de documentação de funcionalidades de software traz para equipes de desenvolvimento. Além disso, objetiva verificar como tal ferramenta pode contribuir para a redução significativa de tempo despendido para a explicação repetitiva de tarefas já executadas anteriormente e como a troca de conhecimento entre membros pode fluir de maneira melhor e com baixo custo.

A ferramenta está prevista de possuir os seguintes requisitos{/}:

\begin{enumerate}
\item Gerenciamento de catálogo de funcionalidades implementadas em um projeto com integração com repositório de código no GitHub
\item Anexo de fontes externas que auxiliem a implementação de determinada funcionalidade
\item Anexo de descrição de etapas da implementação de determinada funcionalidade
\item Recuperação de funcionalidades no catálogo
\item Anexo de variações da implementação de determinada funcionalidade
\end{enumerate}

% Assim, tal ferramenta prevê inicialmente os benefícios de fácil recuperação de referências sem a necessidade de intermédio de colegas de trabalho e, por consequência, garantindo que menos tempo será demandado para a orientação de desenvolvedores.

Este trabalho visa então responder às seguintes perguntas de pesquisa:

\begin{enumerate}
\item Quais são os requisitos para o suporte ferramental em apoio à transferência e reuso do conhecimento organizacional em empresas de desenvolvimento de software?
\item Que outras ferramentas existentes oferecem tal suporte?
\item Como estimular a adoção e o uso desse suporte ferramental em equipes de desenvolvimento de software? 
\item Que outros benefícios e limitações tal suporte provê a equipes de desenvolvimento?

\end{enumerate}

