\chapter{Objetivos}

% gerais e específicos
Este trabalho tem por finalidade analisar que influência uma ferramenta de documentação dinâmica de funcionalidades de software traz para equipes de desenvolvimento. Além disso, objetiva verificar como tal ferramenta pode contribuir para a redução significativa de tempo despendido para a explicação repetitiva de tarefas já executadas anteriormente e como a troca de conhecimento entre membros pode fluir de maneira melhor e com baixo custo.

%falar do o que é uma documentacao dinamica


%falar quais os requisitos e os beneficios 

% perguntas de pesquisa

Assim, este trabalho visa responder às seguintes perguntas de pesquisa:

\begin{enumerate}
\item Quais são os requisitos para o suporte ferramental em apoio à transferência e reuso do conhecimento organizacional em empresas de desenvolvimento de software?
\item Que outras ferramentas existentes oferecem tal suporte?
\item Como estimular a adoção e o uso desse suporte ferramental em equipes de desenvolvimento de software? 
\item Que benefícios e limitações tal suporte provê a equipes de desenvolvimento?

\end{enumerate}

